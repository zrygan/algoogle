\subsection{Binary Search}

Binary search requires \( S \) to be a \textbf{sorted list}: for two indices \( i,j \in \left[0, |S| - 1\right] \) and \( i < j \), the values at those indices follow \( S[i] < S[j] \).

The algorithm starts by calculating the index  of the first element \( l \) (usually \( l = 0 \)), the index of the last element \( h \) (usually \( h = |S| - 1\)), and the midpoint between these two indices \( m = l + \frac{h-1}{2}\). We then check the value at \( m \) and check if \( S[m] = k \). If the comparison is true, then we return \( m \), otherwise we do one of the following:

\begin{itemize}
    \item If \( S[m] > k \), then \( h \gets m - 1 \).
    \item Otherwise, \( l \gets m + 1 \).
\end{itemize}

Then, recalculate \( m \) using the formula above. We continue this process untill \( l = h \) or we get \( S[m] = k \).

\begin{algorithm}
    \caption{\textsc{Binary-Search}}
    \label{alg:binary_search}
    \begin{algorithmic}
        \Require A sorted list \( S \), and a target element \( k \)
        \State \( h \gets |S| - 1 \)
        \State \( l \gets 0 \)
        \While{\( l \leq h \)}
            \State \( m \gets  l + \frac{h-1}{2}\)
            \If{\( S[m] > k \)}
                \State \( h \gets m - 1 \)
            \ElsIf{\( S[m] < k \)}
                \State \( l \gets m + 1 \)
            \Else
                \State \Return m
            \EndIf
        \EndWhile
        \State \Return \( \emptyset \)
    \end{algorithmic}
\end{algorithm}