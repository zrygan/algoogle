\subsection{Linear Search}

In linear search, we determine if \( ((S,k), y) \in R \) by going through each \( s_i \in S \) and checking if \( s = k \). If the equality is true, we return \( i \), otherwise we move to \( s_{i+1} \) and compare again. If we have gone through every \( s_i \) and we are yet to find \( k \), then we have determined that $k \notin S$. Which, in that case, we return \( \emptyset \).

\begin{algorithm}
\caption{\textsc{Linear-Search-Algorithm}}
\label{linear_search}
    \begin{algorithmic}
        \Require A list $S$, and a target element $k$
        \For{$i = 0$ to $|S| - 1$}
            \If{$A[y] = k$}
                \State \Return $k$
            \EndIf
        \EndFor
        \State \Return $\emptyset$
    \end{algorithmic}
\end{algorithm}

We may notice that if \( k = s_1 \), then the algorithm would return in constant time \( \asb{1} \) since the target element is the first element. So, the algorithm would return after one comparison.

However, if \( k = s_{|S| - 1}\), then the algorithm would stop in linear time \( \asb{|S|} \), since it will check every element before getting to the target element. This is also true if \( k \notin S \).

\begin{itemize}
    \item Worst and average case: \( \asb{|S|} \)
    \item Best case: \( \asb{1} \)
\end{itemize}
